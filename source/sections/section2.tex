\newpage
\section{Social Engeneering}

\subsection{First definition}
\begin{itemize}
    \item \textbf{\textit{The science of social phenomena subject to natural and invariable laws, with the goal of discovering these laws.}} - This definition was made in the 1839 by Auguste Comte, inventor if the word "Sociology".
    \item \textbf{\textit{Sociology is the study of human social life, groups and societies.}} - This definition was made by Giddens.
    \item \textbf{\textit{Sociology is the scientific study of society, including the intricate patterns of social behaviour, relationships and human interactions. Is an examination of social institutions, cultural norms and social change using empirical and criticar research and analysis. Those in sociology investigate various aspects of human life, including social stratification, movement and change, with an emphasis on how collective and individual behaviour shapes and is shaped by the broader social context. }} - This is the definition of ChatGPT. 
\end{itemize}
Weber's Wertfreiheit (principle of Value-freedom to be intended as scientific neutrality) states that researchers must refrain from introducing personal value judgments when conducting scientific analysis.
\\From Weberian neutrality to responsibility:
\\don't pretend to be apolitical or indifferent (social research has to clarify risks, trade-offs and consequences helping society make informed decisions) be cause in cybersecurity it's not only about protecting systems but also about protecting society (rights, trust and human dignity), mitigating digital and epistemic asymmetries as well as inequalities.
\subsection{What is social engeneering.}
The use of \textbf{decption} in order to induce a person to divulge private information or unwittingly provide unauthorized access to a computer system or network.
\\ \textbf{Decption} social performances, relies on trust rituals, impression management, symbolic manipulation. We study how people interpret intentions, asess credibility and react under uncertainty.
\\ \textbf{Induce} implies \textbf{persuasion}, \textbf{framing} and \textbf{nudge techniques.}
\\
\newline After all, cybersecurity is fundamentally about an adversarial engagement. Humans must defends machines that are attacked by other humans using machines. So, in addition to the critical traditional fields of computer science, electrical engeneering and mathematics perspectives from other fields are needed. 
\\\textit{Craingen et al., 2014, Defining Cybersecurity.}
\subsection{Ethics and Epistemics}
\begin{itemize}
    \item \textbf{Avalutativity}: The principle of refraining from value judgments in sociological analysis.
    \item \textbf{Extensivity}: A macro-level approach focused on generalisation, stimulus invariance, quantification, and the recognition of limits in broad patterns.
    \item \textbf{Intensivity}: A micro-level approach aimed at understanding the meaning behind actions, qualification of phenomena, and the limits of interpretation.
    \item \textbf{Creative gift of the intellect}: The capacity to generate novel insights and connections beyond immediate observation.
    \item \textbf{The 'Martian' look}: A form of cognitive training that encourages detached observation, as if from an outsider's perspective.
\end{itemize}
\noindent Sociologists are trained to observe both micro- and macro-social phenomena without awe or wonder, even when these phenomena appear distant or disconnected from their own experience.They must avoid taking everyday life and what is considered 'normal' (i.e., institutionalised and seemingly familiar) for granted.The goal is to offer explanations of social phenomena that are less biased and more analytically grounded.In both approaches, the epistemic status of data remains uncertain; information is always partial and context-dependent.
    
\subsection{Basic sociological vocabulary}
\begin{itemize}
    \item  \textbf{Norms:} rules and expectations by which a society guides the behavior of its memmbers.
    \item \textbf{Values:} collective ideas about what is good, desirable and proper.
    \item \textbf{Role:} set of norms, behaviors, and expectations that are associated with a particular social status or position whitin a society. Roles guide how individuals are supposed to act and interact with others specific contexts.
    \item \textbf{Social Structure:} the organized pattern of social relationships and social institutions that together constitute society.
    \item \textbf{Culture:} shared bielifs, values and practices.
\end{itemize}


\subsection{Iceberg Principle}
\noindent The \emph{iceberg principle} in social engineering suggests that the visible part of an attack represents only a small fraction of the overall operation. Most of the preparation, psychological manipulation, and information gathering happens beneath the surface, making the threat far more extensive than what the victim initially perceives.
\begin{figure}[htbp]
    \centering
    \includegraphics[width=0.6\linewidth]{source/images/iceberg.png}
    \caption{Iceberg's principle}
    \label{fig:iceberg_principle}
\end{figure}


\subsection{Overview of Cybersecurity}
\begin{figure}[htbp]
    \centering
    \includegraphics[width=0.6\linewidth]{source/images/cyberoverview.png}
    \caption{Iceberg's principle}
    \label{fig:cyber_overview}
\end{figure}
\paragraph{The name of the "rose".}
\noindent The passage highlights that the absence of clear and shared definitions makes it difficult to measure and classify cybercrime. Barn and Barn argue that without a consistent vocabulary, professionals cannot reliably describe or compare the same phenomena. The reference to Shakespeare’s line about the rose works as a contrast: unlike in poetry, in the study of cybercrime the name truly matters. How we label an act determines whether it is recognised, counted, and addressed. Precise terminology is therefore essential for producing reliable data, shared understanding, and effective responses.
\paragraph{What is?}“A science of cybersecurity offers many opportunities for advances based on 
a multidisciplinary approach, because, after all, cybersecurity is 
fundamentally about an adversarial engagement. Humans must defend 
machines that are attacked by other humans using machines. So, in addition 
to the critical traditional fields of computer science, electrical engineering, 
and mathematics, perspectives from other fields are needed.

\subsection{Cybercrime.} 
\subsubsection{Dichotomic definitions} We could have two different approaches: categorical vs continuum.
\noindent \newline In a Categorical Approach we could distinguish
\begin{itemize}
    \item \textbf{Cyber-enabled:} are traditional crimes that predate the advent of the technology, and are now facilitated or have been made easier (i.e., enabled) by cyber technology. Crimes range from white-collar crime to drug trafficking, to online harassment, terrorism and beyond. 
    \item \textbf{Cyber-dependent:}  are crimes that arose with the advent of technology and cannot exist (i.e., dependent) outside of the digital world, e.g., hacking, such as ransomware attacks or hacktivism
\end{itemize}
\noindent In a Continuum Approach
\begin{itemize}
    \item \textbf{Type I:} cybercrimes are considered to be more technical in nature, for example, hacking, similar to ‘cyberdependent’ crimes as described previously.
    \item \textbf{Type II:} are generally considered to involve more human contact, for example, online gambling, similar to ‘cyberenabled’ crimes as described above.
\end{itemize}

\subsubsection{Trichotomic definitions}
Cybercrime is commonly divided into three categories:
\begin{itemize}
  \item \textbf{Crimes Against the Machine} (computer integrity crimes): hacking, cracking, DoS/DDoS.
  \item \textbf{Crimes Using the Machine} (computer-assisted crimes): piracy, fraud, robbery.
  \item \textbf{Crimes In the Machine} (computer content crimes): hate speech, harassment, pornography.
\end{itemize}

\paragraph{Cybersecurity Foundations}
Cybersecurity is a contested, multidisciplinary field involving adversarial dynamics:
\begin{itemize}
  \item Humans defend machines from other humans using machines.
  \item Requires input from computer science, engineering, law, philosophy, and sociology.
  \item Incidents often arise from misalignment between \textit{de jure} (legal) and \textit{de facto} (actual) property rights.
\end{itemize}

\paragraph{Some definitions.} Here the definitions by Craigen et al. 2014.\\
\begin{table}[h!]
\centering
\begin{tabular}{|c|l|l|}
\hline
\# & Focus & Keywords \\
\hline
1 & Intrusion detection & defensive methods \\
2 & Network protection & malicious damage \\
3 & Holistic toolkit & policies, safeguards \\
4 & Cyber-attack defense & cyberspace protection \\
5 & Legal framing & unauthorized use \\
6 & System integrity & modification, exploitation \\
7 & National continuity & infosociety, infrastructure \\
8 & CIA principles & confidentiality, integrity, availability \\
9 & Risk reduction & authentication, encryption \\
10 & Property rights & de jure vs de facto misalignment \\
\hline
\end{tabular}
\end{table}

\subsection{What is cybersecurity?}
Cybersecurity is the organization and collection of resources, processes, and structures used to protect cyberspace and cyberspace-enabled systems from occurrences that misalign de jure from de facto property rights.
\\The gap that cybercrime makes, broadly construed, is an occurrence the main result of which is to make a gap (misalignment) between a de-jure right and a factual situation.For example ransomwares jeopardize access and property of data (asset).Still people sometimes fail to catch that conduct in the cyberspace has real implications in the analogical world. But digital is not detached from the “real world”. Laws from the physical world are not suspended online.
\subsection{Paradigms and Dimensions}

\begin{center}
\begin{itemize}
  \item \textbf{Continuity}: preserving the digital society.
  \item \textbf{Ecosystem}: human-system interactions.
  \item \textbf{Risk}: managing uncertainty.
  \item \textbf{Ownership}: access, control, exclusion.
  \item \textbf{Unpredictability}: threats can be accidental or unknown.
\end{itemize}
\end{center}

\paragraph{Cybercrime Categories (Extended Table)} The table summarizes the five-category typology of cybercrimes.\\
\begin{table}[h!]
\centering 
\begin{tabular}{|l|p{8cm}|}
\hline
\textbf{Category} & \textbf{Subtypes} \\
\hline
Cyberbullying \& Stalking & Denigration, Exclusion, Flaming, Harassment, Outing, Cyberstalking, Dating abuse \\
Digital Piracy & Piracy \\
Hacking \& Malware & Unauthorized access, viruses, file destruction, service theft, credit card fraud, malware \\
Identity Theft & Identity fraud \\
Sex-Related Crimes & Grooming, sexting, CSAM, revenge porn, sextortion \\
\hline
\end{tabular}
\end{table}

\subsection{Limits and key challenges}
There is a need to scrutinize the evolving landscape of technology that brings with it new cybercriminal behaviors. Need for further empirical studies regarding the criminal use of advanced technologies such as AI, Machine/Deep Learning, Deep Fakes and Virtual Reality, which are relatively unaccounted for by current classification frameworks, as well as the use of technology for terror-related activities including extremism and radicalization.
\begin{figure}[htbp]
    \centering
    \includegraphics[width=0.6\linewidth]{source/images/limit_cyber.png}
    \caption{Iceberg's principle}
    \label{fig:cyber_limit}
\end{figure}