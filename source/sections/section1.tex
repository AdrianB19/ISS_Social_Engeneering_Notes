\newpage
\section{Introduction}

\subsection{Exam Informations}
This part of the course is weighted about the 25\% the entire exam.
The exam of this module includes contributions from two type of questions.
\begin{itemize}
    \item \textbf{Theoretical Knowledge:} you have to evaluate what you learned about sociological theory relevant to cybersec. In particular you have to interpret / comment a definition, classification scheme or sociological concept ("iceberg", "Cialdini's principle", "3 common traits"). Avoid common sense and demonstrate rigorous and logical understanding of the sociological material, using of precise terminology (e.g. "epistemic asymmetry","technocratic dominance","teleological replacement","norms, values, roles").
    \item \textbf{Application to Practical Case(s):} focus on apply theory to a real-world scenario or scenatio-based question. You have to identify latent fators in social engeneering (urgency, authority, impersionation, etc.) and reference relevant theoretical ideas (e.g. Cialdini's principles, sociological definitions).
\end{itemize}
An answering strategy is to be structured and coincise to connetc the general theoretical knowledge to the specific question.

\subsection{Goal of the course}
This module is complementary to the main part of the class, which is strongly tech-oriented. You will identify sociotechnical vulnerabilities and help mitigate their consequences.The technical aspect is necessary but not sufficient for cybersecurity, the humans are structural points of every security system, but not only weakpoints.This part will teach to analyze complex social mechanism, such as the social construction of knowledge, risk, trust, manipulation and communication.
